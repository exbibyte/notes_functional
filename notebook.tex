\documentclass[8pt]{extarticle}
 
%% \usepackage[fleqn]{amsmath}
\usepackage[margin=1in]{geometry}
\usepackage{amsmath,amsfonts,amsthm,bm}
\usepackage{breqn}
\usepackage{amsmath}
\usepackage{mathtools}
\usepackage{amssymb}
\usepackage{bbm}
\usepackage{tikz}
\usepackage[ruled,vlined,linesnumbered,lined,boxed,commentsnumbered]{algorithm2e}
\usepackage{siunitx}
\usepackage{graphicx}
\usepackage{subcaption}
%% \usepackage{datetime}
\usepackage{multirow}
\usepackage{multicol}
\usepackage{mathrsfs}
\usepackage{fancyhdr}
\usepackage{fancyvrb}
\usepackage{parskip} %turns off paragraph indent
\pagestyle{fancy}

\usepackage{xcolor}
\usepackage{mdframed}

\usepackage[small]{titlesec}

\usepackage{hanging}

\usetikzlibrary{arrows}

\DeclareMathOperator*{\argmin}{argmin}
\newcommand*{\argminl}{\argmin\limits}

\newcommand{\mathleft}{\@fleqntrue\@mathmargin0pt}
\newcommand{\R}{\mathbb{R}}
\newcommand{\Z}{\mathbb{Z}} 
\newcommand{\N}{\mathbb{N}}
\newcommand{\ppartial}[2]{\frac{\partial #1}{\partial #2}}
\newcommand{\p}{\partial}
\newcommand{\te}[1]{\text{#1 }}
\newcommand{\norm}[1]{\|#1\|}

\setcounter{MaxMatrixCols}{20}

% remove excess vertical space for align equations
\setlength{\abovedisplayskip}{0pt}
\setlength{\belowdisplayskip}{0pt}
\setlength{\abovedisplayshortskip}{0pt}
\setlength{\belowdisplayshortskip}{0pt}

\newtheorem{theorem}{Theorem}[section]
\newtheorem{corollary}{Corollary}[theorem]
\newtheorem{lemma}[theorem]{Lemma}

% \newtheorem{mdtheorem}{Theorem}
% \newenvironment{theorem}
% {\begin{mdframed}[
%   backgroundcolor=green!10,
%   topline=false,
%   rightline=false,
%   bottomline=false,
%   leftline=false
%   ]\begin{mdtheorem}}
%     {\end{mdtheorem}\end{mdframed}}

\begin {document}

\lhead{Notes - Functional Programming, Yuan Liu}

% \begin{align}
%   \nabla f_{k+1}^T p_k & \geq c_2 \nabla f_k^T p_k\\
%   \frac{\partial \phi(\alpha_k)}{\partial \alpha_k} & \geq \frac{\partial \phi(0)}{\partial \alpha_k}\\
%   &c_1,\alpha_k \in (0,1)\\
%   &0<c_1<c_2<1
% \end{align}
  
\begin{multicols*}{2}

  Gathered notes from:
  \begin{itemize}
  \item Haskell Programming from First Principles \cite{hpffp}
  \end{itemize}

  \section{Monoid}
  todo

  \vfill\null
  \columnbreak

  \section{Functor}
  todo

  \vfill\null
  \columnbreak

  \section{Applicative}
  todo

  \vfill\null
  \columnbreak

  \section{Monad}
  todo

  \vfill\null
  \columnbreak
    
  \section{Arrow operator as Functor and Applicative}

  \subsection{Functor}

  \verb% ((->) r) = (r ->)% as a functor\\
  \verb% (r ->) *% expects a type as argument\\
  instances of \verb% (->) r *% as a type class\\
  examples of other functors: \verb%[] *%, \verb%Maybe *%\\
  functor as a type constructor
  
  fmap:\\
  \verb%<$>:: (a -> b) -> F a -> F b%\\
  let \verb%F = (->) r%\\
  \verb%<$>:: (a -> b) -> ((->) r) a -> ((->) r) b%\\
  \verb%<$>:: (a -> b) -> (r -> a) -> (r -> b)%\\
  \verb%<$>:: (a -> b) -> (r -> a) -> r -> b%

  composition operator:\\
  \verb% (.) :: (b -> c) -> (a -> b) -> a -> c%

  therefore,\\
  \verb%<$> = (.)% where \verb%F = (->) r% for functor

  \subsubsection{example}
  \verb%(+) <$> (*2)%\\
  \verb%(+) . (*2)%\\
  \verb%\x -> (+) ((*2) x) %\\
  \verb%\x -> (+) (x*2) %\\
  \verb%\x -> x*2 :: a -> a%\\
  \verb%(\x -> (+) (x*2)) :: a -> (a -> a)%\\
  \verb%(\x -> (+) (x*2)) :: a -> a -> a%\\
  \verb%(\x -> ((x*2)+) :: a -> a -> a%\\
  \verb%(\x -> (\y -> (x*2) + y) :: a -> a -> a%
  
  \subsection{Applicative}
  apply:\\
  \verb%<*>:: F (a -> b) -> F a -> F b%\\
  let \verb%F = (->) r = r ->%, then\\
  \verb%<*>:: ((->) r) (a->b) -> ((->) r) a -> ((->) r) b%\\
  \verb%<*>:: (r -> a -> b) -> (r -> a) -> (r -> b)%\\

  \verb%pure :: a -> F a%\\
  \verb%pure x = ((->) r) x = r -> x :: F a%

  \subsubsection{example}
  \verb%(+) <$> (*2) <*> (+10)%\\
  \verb%(+) . (*2) <*> (+10)%\\
  \verb%(\x -> (+) (x*2)) <*> (\x -> x + 10)%\\
  \verb%\x -> (+) (x*2) (x+10)%

  types:\\
  \verb%\x -> :: (->) r%\\
  \verb%(+) (x*2) :: a -> b% where \verb%x% is fixed\\
  \verb%\x -> (+) (x*2) :: ((->) r) a -> b%\\
  
  \verb%\x -> x + 10 :: r -> a  = ((->) r) a%\\
  
  \verb%(+) (x*2) (x+10) :: b% where \verb%x% is fixed\\
  \verb%\x -> (+) (x*2) (x+10) :: ((->) r) b%\\
  
  \verb%<*> :: (((->) r) a -> b) -> (((->) r) a) -> (((->) r) b)%

  thus types are as expected for applicative
  
  \vfill\null  
  \pagebreak
    
  \bibliographystyle{plain}
  \bibliography{notebook}
    
  \end{multicols*}

\end {document}
